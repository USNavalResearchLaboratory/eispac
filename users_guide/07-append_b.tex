
\chapter{Appendix B: Useful EISPAC doc strings}

\section{fit\_spectra()}
\label{sec:fitspec}

\begin{lstlisting}[language=Python]
def fit_spectra(inten, template, parinfo=None, wave=None, errs=None,
                min_points=10, ncpu='max', skip_fitting=False):
    """Fit one or more EIS line spectra using mpfit (with multiprocessing).

    Parameters
    ----------
    inten : EISCube object, array_like, or filepath
        One or more intensity profiles to be fit. The code will loop over 
        the data according to its dimensionality. 3D data is assumed to be a 
        full EIS raster (or a sub region), 2D data is assumed to be a single 
        EIS slit, and 1D data is assumed to be a single profile.
    template : EISFitTemplate object, dict, or filepath
        Either an EISFitTemplate, a 'template' dictionary, or the path to a
        template file.
    parinfo : list, optional
        List of dictionaries with fit parameters formatted for use with mpfit.
        Will supercede any parinfo lists loaded from an EISFitTemplate. 
        Required if the 'template' parameter is given as a dictionary.
    wave : array_like, optional
        Associated wavelength values for the spectra. Required if 'inten' is
        given as an array and ignored otherwise.
    errs : array_like, optional
        Intensity error values for the spectra. Required if 'inten' is given 
        as an array and ignored otherwise.
    min_points : int, optional
        Minimum number of good quality data points (i.e. non-zero values & 
        errs) to be used in each fit. Spectra with fewer data points will be 
        skipped. Default is 10.
    ncpu : int, optional
        Number of cpu processes to parallelize over. Must be less than or 
        equal to the total number of cores the system has. If set to 'max' or 
        None, the code will use the maximum number of cores available. 
        Default is 'max'.
        Important: due to the specifics of how the multiprocessing library 
        works, any statements that call fit_spectra() using ncpu > 1 MUST be 
        wrapped in a "if __name__ == __main__:" statement in the top-level 
        program. If such a "name guard" statement is not detected, this 
        function will fall back to using a single process.
    skip_fitting : bool, optional
        If set to True, will skip the fitting altogether and just return an 
        empty EISFitResult instance. Used mainly for testing.

    Returns
    -------
    fit_res : EISFitResult class instance
        An EISFitResult object containing the output fit parameters.
    """
\end{lstlisting}

\section{EISFitResult methods}
\label{sec:EISFitResult}
\begin{lstlisting}[language=Python]
def get_params(self, component=None, param_name=None, coords=None,
               casefold=False):
    """Extract parameters values by component number, name, or pixel coords

    Parameters
    ----------
    component : int or list, optional
        Integer number (or list of ints) of the functional component(s).
        If set to None, will return the total combined fit profile.
        Default is None.
    param_name : str, optional
        String name of the requested parameter. If set to None, will not
        filter based on paramater name. Default is None
    coords : list or tupple, optional
        (Y, X) coordinates of the requested datapoint. If set to None, will
        instead return the parameters at all locations. Default is None
    casefold : bool, optional
        If set to True, will ignore case when extracting parameters by
        name. Default is False.

    Returns
    -------
    param_vals : numpy array
        Parameter values
    param_errs : numpy array
        Estimated parameter errors
    """
\end{lstlisting}

\begin{lstlisting}[language=Python]
def get_fit_profile(self, component=None, coords=None, num_wavelengths=None):
    """Calculate the fit intensity profile (total or component) at a location.

    Parameters
    ----------
    component : int or list, optional
        Integer number (or list of ints) of the functional component(s).
        If set to None, will return the total combined fit profile.
        Default is None.
    coords : list or tupple, optional
        (Y, X) coordinates of the requested datapoint. If set to None, will
        instead return the parameters at all locations. Default is None
    num_wavelengths : int, optional
        Number of wavelength values to compute the fit intensity at. These
        values will be equally spaced and span the entire fit window. If set
        to None, will use the observed wavelength values. Default is None.

    Returns
    -------
    fit_wave : numpy array
        Wavelength values
    fit_inten : numpy array
        Fit intensity values
    """
\end{lstlisting}
